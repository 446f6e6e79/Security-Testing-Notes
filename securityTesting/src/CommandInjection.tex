\section{Command Injection}
Command injection is an attack in which an attacker's goal is the \textbf{arbitrary execution of commands} on the \textbf{host operating system}.

\paragraph{Problem:} Untrusted external data is directly passed to a command interpreter. If the data is crafted to include commands, those commands may be executed and the interpreter may be forced to perform actions beyond its intended function.

\paragraph{Underlying conditions:} For command injection to succeed, the application typically meets three main conditions:
\begin{itemize}
    \item the application has the \textbf{privileges/permissions} to execute system commands;
    \item the application uses \textbf{user-provided data} as part of a system command;
    \item the user-provided data is not properly \textbf{escaped} or \textbf{sanitized} before use.
\end{itemize}

\paragraph{Mitigation:}
\begin{itemize}
    \item Avoid invoking shell/OS execution functions when possible;
    \item Treat all external input as untrusted: validate, normalize, and apply \textbf{whitelisting} where feasible;
    \item Use \textbf{parameterized APIs} or pass user input as separate arguments rather than concatenating it into command strings.
    \item Run the application with the \textbf{least privileges} necessary to reduce the impact of a successful attack.
\end{itemize}