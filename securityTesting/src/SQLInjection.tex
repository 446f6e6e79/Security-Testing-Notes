\section{SQL Injection}
\label{sec:sql-injection}
\textit{SQL injection} is a vulnerability that allows an attacker to inject unintended SQL code into queries executed by an application. 
Any software that constructs SQL queries using external input may be vulnerable to this type of attack.
\subsection{Underlying idea}
The application executes an SQL query where some parameters are taken from user input. For example:
\begin{lstlisting}[language=SQL]
SELECT * FROM users WHERE userId = <user_id> AND password = <user_password>;
\end{lstlisting}
If the application naively substitutes user-supplied values into the query, without proper validation or sanitization, an attacker can craft input that changes the query semantics.

\subsection{Example attack}
Suppose the application builds the query by concatenation of the user input. If an attacker supplies:
\begin{verbatim}
user_id: ' OR '1'='1
user_password: ' OR '1'='1
\end{verbatim}
the resulting query becomes:
\begin{lstlisting}[language=SQL]
SELECT * FROM users
    WHERE userId = '' OR '1'='1'
    AND password = '' OR '1'='1';
\end{lstlisting}
Because the boolean expression \texttt{'1'='1'} is always true, the WHERE clause is satisfied for all rows, allowing the attacker to bypass authentication or access data they shouldn't.

\subsection{SQL Injection type}
There are two main types of SQL injection techniques:
\subsubsection{In-Band SQL injection:}
In this technique, also called CLASSICAL, the attacker uses the same way to hack the database and get the data. The attacker has the possibility to modify the original query and receive the result from the database. This is composed of 3 variants:
\begin{itemize}
    \item \textbf{Classic SQL Injection}: basic and traditional version, seen in the previous example;
    \item \textbf{Error-Based SQL Injection}: the attacker makes the database produce error messages that can help to gather information about the database structure.
    \item \textbf{Union-Based SQL Injection}: the attacker uses the \texttt{UNION} SQL operator to combine the results of the original query with the results of a malicious query.
\end{itemize}

\subsubsection{Inferential SQL Injection:}
This type of injection, also called BLIND, does not show any error message. It's more difficult to exploit, as it returns information
only when the application is given SQL payloads that return true or false responses from the server.
By observing the responses, an attacker can extract sensitive information.

\begin{itemize}
    \item \textbf{Boolean Based}: the attacker observes the behavior of the database server and the application, after combining legitimate queries with malicious data, using boolean operators.
    For example:
    \begin{lstlisting}[language=SQL]
    SELECT * FROM users WHERE userId = '1' AND '1'='2'; -- Expected to return no results as '1'='2' is false
    SELECT * FROM users WHERE userId = '1' AND '1'='1'; -- Could return results as '1'='1' is true
    \end{lstlisting}
    If we see a different behavior between the two queries, we can infer that the userId '1' exists in the database.
    \item \textbf{Time Based}: the attacker observes the behavior of the database server and the application, after combining legitimate queries with SQL commands that cause time delays.
    For example, we could use the following payloads:
    \begin{lstlisting}[language=SQL]
    SELECT * FROM users WHERE userId = '1' AND sleep(10); -- Introduces a delay of 10 seconds
    \end{lstlisting}
    If the application takes longer to respond, we can infer that the userId '1' exists in the database.
    \item \textbf{Out-of-Band}: this technique is used when the application show the same behavior regardless of the query executed. The attacker uses other channels to retrieve the output of the query, such as HTTP requests. 
\end{itemize}
\subsection{SQL Injection mitigation}
To prevent SQL injection attacks, developers should implement the following best practices:
\begin{itemize}
    \item Escaping all user supplied input. This means to treat all the data as plain text, rather than executable code or other harmful inputs;
    \item Define a list of allowed inputs, to ensure only them are executed;
    \item Use of prepared statements with parametrized queries. Programming languages allow to use  predefined structures to compose SQL queries by accepting only values of specified types.
    This allow to separate code from data, preventing the execution of malicious SQL code.
\end{itemize}