\section{Code Weaknesses and Vulnerabilities}
We now present a selection of the \textbf{most dangerous software weaknesses}, to illustrate common coding patterns that lead to security vulnerabilities and to guide analysis and testing efforts.
Knowing this classification can help us with:
\begin{itemize}
    \item \textbf{conducting an effective testing activity}: we know what to search while testing, so we can prioritize some issues, over others;
    \item \textbf{improving the development process}: know solutions (ex: patterns, guidelines) to help developers improve their software, avoiding the introduction of weaknesses;
    \item \textbf{driving bug-fixing}: prioritize the issues to solve, helping also to understand how to fix it.
\end{itemize}
\subsection{Common Weakness Enumeration (CWE)}
\subsubsection{Use-After-Free(CWE-416)}
The software \textbf{reuses} or \textbf{references memory}, after it has been \textbf{freed}. At some point afterwards, the memory may be allocated to another pointer. Any operation using the original pointer is no longer valid.

\begin{lstlisting}[language=C]
    char* ptr = (char*)malloc (SIZE);
    if(err){
        abrt = 1;
        free(ptr);
    }
    if(abrt){
        //Assessing a pointer 
        logError("operation aborted before commit", ptr);
    }
\end{lstlisting}

\textbf{UAF} can be found also when other resources, such as \textbf{file} or \textbf{network connections} are \textbf{mismanaged}.

\paragraph{Prevention:} Once freed, all pointers should be set to \textbf{NULL}.

\paragraph{Detection:} Using \textbf{static analysis} or \textbf{Fuzzing}.

\subsubsection{Heap-based Buffer Overflow(CWE-122)}
A particular case of the \textbf{buffer overflow}, where the buffer is allocated in the \textbf{heap} portion of the memory. This means that the buffer was allocated using \textbf{malloc()}.

\begin{lstlisting}[language=C]
    #define BUFSIZE 4
    int main(int argc, char **argv) {
        char *buf;
        buf = (char *)malloc(sizeof(char)*BUFSIZE);
        strcpy(buf, argv[1]);
    }
\end{lstlisting}
The problem here is caused by the function \verb|strcpy()|, which has no limits on the length of the element that should be copied to the buffer.

\paragraph{Prevention:} Other equivalents, but \textbf{safer} functions should be used. For example \verb|strncpy()| takes as an argument the maximum number of characters that should be copied.
\\Also, automatic buffer overflow detection mechanisms can be used.

\paragraph{Detection:} Using \textbf{static analysis} or \textbf{Fuzzing}.

\subsubsection{Out-of-bounds Write(CWE-787)}
The software writes data past the end, or before the beginning of the intended buffer.

\begin{lstlisting}[language=C]
    int id_sequence[3];
    
    id_sequence[0] = 123;
    id_sequence[1] = 234;
    id_sequence[2] = 345;
    id_sequence[3] = 456;
\end{lstlisting}

\paragraph{Prevention:} Always verify that the buffer is as large as specified and that its boundaries are respected.

\paragraph{Detection:} Using \textbf{static analysis} and \textbf{dynamic analysis}.

\subsubsection{Improper Input Validation(CWE-20)}
The software receives input values, but it \textbf{does not validate} or \textbf{incorrectly validates} them. In case of no-validation, unintended input values can \textbf{alter the program control flow}.

\begin{lstlisting}[language=C]
    public static final double price = 20.00;
    int quantity = currentUser.getAttribute("quantity");
    double total = price * quantity;
    chargeUser(total);
\end{lstlisting}
In this example, if the user inputs a negative number for the quantity, it can lead to an account credit instead of a debit.

\paragraph{Prevention:} Always assume that all inputs are malicious. An input validation framework can be used to check all the untrusted inputs.

\paragraph{Detection:} Using \textbf{static analysis} and \textbf{dynamic analysis}

\subsubsection{Improper Neutralization of Special Elements used in an OS Command(CWE-78)}
The software constructs an \textbf{OS command} using an \textbf{external input}, but it does not neutralize special elements that could modify the intended OS command.

\begin{lstlisting}[language=php]
    $userName = $_POST["user"];
    $command = 'ls -l /home/' . $userName;
    system($command);
\end{lstlisting}
This code takes the name of a user and lists the content of the user directory. There is no check on the variable \verb|$userName|, that could contain an \textbf{arbitrary OS command} such as: \verb|“;rm -rf /”|.
Since the \verb|;| works as a command separator in Linux, such input would delete the entire file system.

\paragraph{Prevention:} Use library calls rather than external processes and try to block the user from using semi-colons or other special characters.

\paragraph{Detection:} Using \textbf{dynamic analysis}

\subsubsection{De-serialization of Untrusted Data(CWE-502)}
The software deserializes untrusted data without verifying that the resulting data will be valid.

\begin{lstlisting}[language=Java]
    try {
        File file = new File("object.obj");
        ObjectInputStream in = new ObjectInputStream(new FileInputStream(file));
        javax.swing.JButton button = (javax.swing.JButton) in.readObject();
        in.close();
    }
\end{lstlisting}
The code deserializes an object from a file, received from an UI button and does not verify the source or contents of the received file.

\paragraph{Prevention:} When deserializing data, a new Object should be populated, rather than just deserializing it.

\paragraph{Detection:} Using \textbf{static analysis}

\subsubsection{Server-Side Request Forgery(CWE-918)}
\label{Server-Side Request Forgery}
A security vulnerability in a \textbf{server A} allows an attacker to cause the \textbf{server-side application} to make requests to an unintended location (i.e., \textbf{another server B} or service). 
\\The \textbf{attacker’s request} is \textbf{sent from the back end of the server}, so the \textbf{target service believes it came from a legitimate source}. 
Attackers can \textbf{bypass access controls} (i.e., firewalls) that \textbf{prevent direct attacks on the target server} by exploiting trusted relationships between servers. 
The compromised server can also be used as a proxy to conduct port scanning of internal hosts or to access restricted URLs and documents.
\begin{comment}
    Add image here!!!
\end{comment}
\paragraph{Prevention:} 
\begin{itemize}
    \item Limit outbound traffic from the application server.
    \item Build a whitelist of trusted domains and IP addresses.
    \item Sanitize user input to prevent potential risks.
\end{itemize}

\paragraph{Detection:} Automated static analysis

\subsubsection{Access of Resource Using Incompatible Type(CWE-843)}
The software initializes a resource such as a pointer, object or variable, using \textbf{one type}, but later accesses that resource using a type that is incompatible with the original type. Type confusion can lead to out-of-bounds memory access.

\begin{lstlisting}[language=php]
    $value = $_GET['value'];
    $sum = $value + 5;
    echo "value parameter is '$value'<p>";
    echo "SUM is $sum";
\end{lstlisting}
An attacker could supply an input string such as \verb|value[] = 123|. From that point, value is treated as an array type, causing an error when the sum is calculated.

\paragraph{Prevention:} Attention to implicit and explicit type conversion.

\paragraph{Detection:} Using \textbf{static analysis}

\subsubsection{Improper Limitation of a Pathname to a Restricted Directory(CWE-22)}
The product uses an \textbf{external input} to \textbf{construct a pathname} for a file or directory, located into a restricted parent directory. Without the neutralization of special elements, the path can relate to a different location, not supposed to be accessible.

\begin{lstlisting}[language=Java]
    String path = getInputPath();
    if (path.startsWith("/safe_dir/")){
        File f = new File(path);
        f.delete()
    }
\end{lstlisting}
\paragraph{Prevention:} 
\begin{itemize}
    \item Always assume all input is malicious;
    \item When the set of acceptable object, such as filenames or URLs is limited and known, create a mapping from a set of \textbf{fixed input values} to the \textbf{actual} \textbf{filenames} or \textbf{URLs}, rejecting all the other inputs.
\end{itemize}
\paragraph{Detection:} Using \textbf{static analysis} or \textbf{dynamic analysis}.

\subsubsection{Missing Authentication for Critical Function}(CWE-306)
The product \textbf{does not perform any authentication} for a functionality that requires a provable user identity.
\paragraph{Prevention:} Where possible, avoid using custom authentication routines. Instead, consider using authentication capabilities as provided by libraries or frameworks.

\paragraph{Detection:} Using \textbf{static analysis} or \textbf{dynamic analysis}, such as \textbf{vulnerability scanning}.

\subsection{Most common security risks}
\textbf{OWASP} is an open source project, providing guides, guidelines and suggestions to develop and test secure applications. Its Top-10 lists the most critical security risks for web applications.

\subsubsection{Broken Access Control (A01-2021)}
The \textbf{access control} enforces policy such that \textbf{users cannot act outside their intended permissions}. Its failure leads to \textbf{unauthorized information} disclosure, modification, or destruction.

\paragraph{Example:} The application uses unverified data in SQL to access information:

\begin{lstlisting}[language=Java]
//Trying to access using this url
//https://example.com/app/accountInfo?acct=myacct    
PreparedStatement pstmt = connect.prepareStatement(query);
pstmt.setString(1, request.getParameter("acct")); 
//myacct is a "reference" to an account
ResultSet results = pstmt.executeQuery( );
\end{lstlisting}
If an attacker modifies the browser's \textbf{acct parameter} with whatever account number they want, the attacker can access any other user's account.
\paragraph{Prevention:}
\begin{itemize}
    \item Deny by default;
    \item Access control is effective only in truster server-side code;
    \item \textbf{Access control unit should be included in the tests}.
\end{itemize}

\subsubsection{Cryptography Failures (A02-2021)}
Violation of the protection needed for exchanged data. Data can't be transmitted in clear text, but should always be encrypted, using properly set cryptographic algorithms.
\paragraph{Example:} An application correctly encrypts credit card numbers in a database before storing them. Data is automatically decrypted once retrieved, allowing an SQL injection to retrieve all the information in clear text.
\paragraph{Prevention:}
Proper encryption, with a proper setting of the used cryptographic mechanism.

\subsubsection{Injection (A03-2021)}
\paragraph{Description:}
An application is vulnerable to this type of attack when user input data is not validated by the application before using it as command parameters.
\paragraph{Example:}   
\begin{lstlisting}[language=Java]
String query = "SELECT \* FROM accounts WHERE custID='" +
    request.getParameter("id") + "";
\end{lstlisting}
If the attacker manages to set the id parameter for example to \textit{‘ or ‘1’=’1} the query will return all the records from the account table.
\paragraph{Prevention:} Use server-side input validation, checking in particular for escape characters.

\subsubsection{Insecure design (A04-2021)}
Insecure design concerns weaknesses related to missing or ineffective control design. An usual example could be the use of "questions and answers" for the credential recovery workflow. They in fact cannot be trusted as evidence of user identity.
\paragraph{Prevention:} Always adopt a secure design methodology.

\subsubsection{Security Misconfiguration (A05-2021)}
\paragraph{Description:}
Refers to a variety of situations, such as:
\begin{itemize}
    \item Missing appropriate security hardening or improperly configured permissions;
    \item Unnecessary features installed (ex: ports, accounts, privileges);
    \item Default account, with default passwords;
    \item Missing error handling, revealing stack traces or other important information;
    \item Out of date software.
\end{itemize}
\paragraph{Prevention:}
A secure installation process should be implemented. Always review and update configuration.

\subsubsection{Vulnerable and Outdated Components (A06-2021)}
The component of a software can be vulnerable, unsupported, or out of date.
\paragraph{Prevention:} 
\begin{itemize}
    \item Remove unused dependencies, unnecessary features, components and documentation;
    \item Continuously monitor the versions of both client and server components and their dependencies;
    \item Obtain components only from official sources;
\end{itemize}

\subsubsection{Identification and Authentication Failures (A07-2021)}
\textbf{Authentication}, \textbf{confirmation} of user's \textbf{identity} and \textbf{session management} are critical to protect against authentication-related attacks.
\paragraph{Prevention:}
\begin{itemize}
    \item Implement proper identification and authentication processes (ex: MFA, to prevent brute force, stolen credential reuse attacks);
    \item Do not permit the usage of weak passwords;
    \item Use a \textbf{server-side}, secure, \textbf{session manager}.
\end{itemize}

\subsubsection{Software and Data Integrity Failures (A08-2021)}
\paragraph{Description:}
\paragraph{Prevention:}

\subsubsection{Security Logging and Monitoring Failures (A09-2021)}
\paragraph{Description:}It relates to the logging and monitoring capability to detect security breaches. You will make yourself vulnerable to information leakage by making logging and alerting events visible to the users.
\paragraph{Prevention:} Ensure proper logging and monitoring.

\subsubsection{Server-Side Request Forgery (A10-2021)}
Already talked about at \ref{Server-Side Request Forgery}.