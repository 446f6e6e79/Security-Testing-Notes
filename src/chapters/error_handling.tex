\section{Error Handling}
Error handling is a mechanism used to manage errors that occur during the execution of a program. 
It allows the application to recover from unexpected conditions and avoid crashes. 
Proper error handling is an essential part of an application's overall security.
\\\\Most of the attacks always begin with reconnaissance activities, where the attacker tries to gather
as much technical information as possible about the target system.

\paragraph{Problem:}
Improper error handling occurs when software fails to correctly handle certain error conditions, leaving the program in an insecure or unstable state.  
This can lead to:
\begin{itemize}
    \item \textbf{Program crashes:} unhandled exceptions or errors can cause the program to terminate unexpectedly, leading to a denial of service;
    \item \textbf{Information leakage:} detailed error messages such as stack traces may reveal sensitive information about the application's internal workings.
\end{itemize}
Improper error handling and unhandled exceptions can be exploited by attackers to collect useful information.

\paragraph{Mitigation:}
\begin{itemize}
    \item Always handle errors and exceptions appropriately, considering the specific context of each case;
    \item Avoid displaying detailed error messages to end users — log them securely instead;
    \item Do not silently ignore or mask exceptions, as this can hide critical failures and lead to unpredictable behavior;
    \item Always check the return value or status code of functions and system calls for possible errors;
\end{itemize}