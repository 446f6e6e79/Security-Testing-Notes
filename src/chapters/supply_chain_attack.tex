\section{Supply Chain Attack}
To understand supply chain attacks, it's essential to grasp the concept of dependencies in software development.
\subsection{Dependencies in Software Development}
In modern software development, applications often rely on external libraries, frameworks, and tools to enhance functionality and to avoid reinventing the wheel. 
These external components are known as dependencies, and they are usually managed through package managers (like npm, Maven, pip, etc.).
\\Dependencies can be:
\begin{itemize}
    \item \textbf{Direct dependencies}: libraries or packages that are explicitly included in a project's codebase. We are the owners of these dependencies;
    \item \textbf{Transitive dependencies}: libraries that are not directly included but are dependencies of the direct dependencies.
\end{itemize}
Dependencies can introduce vulnerabilities in our software, if they contain security flaws or if they are compromised.

\subsection{What is a Supply Chain?}
A software supply chain refers to the entire ecosystem of components, tools, and processes involved in the development, distribution, and maintenance of software applications.
Usually, the inventory of a software supply chain is declared in a \textbf{Software Bill of Materials} (SBOM).

\subsection{Supply Chain Attacks}
A supply chain attack targets third-party components or services that are part of the software supply chain, to compromise a target application.
They are indirect, in the sense that attackers do not directly attack the target application, but rather exploit vulnerabilities in the components or services it relies on.

\subsubsection*{Mechanism of Supply Chain Attacks}
The typical steps of a supply chain attack are:
\begin{itemize}
    \item The attacker identifies a \textbf{software vendor} to add malicious code to their product;
    \item The malicious code is then distributed to the vendor's customers through legitimate software updates;
    \item When the target application updates or installs the compromised component, the malicious code is executed within the target environment.
\end{itemize}

\subsubsection*{Attack Vectors}
Since this is an indirect attack, the attack vectors are different from traditional attacks.
Some common attack vectors include:
\begin{itemize}
    \item \textbf{Source code}: attackers compromise the proprietary code. Usually the credentials of developers as well as the access to the source code repository are needed;
    \item \textbf{Developer accounts}: attackers gain access to developer accounts to inject malicious code into the software during development or distribution;
    \item \textbf{Dependencies}: attackers compromise third-party libraries or packages that are used as dependencies in the target application;
    \item \textbf{Build tooling}: attackers can compromise build tools or CI/CD pipelines to inject malicious code during the build process;
    \item \textbf{Updates}: attackers compromise the software update mechanism, distributing malicious updates to users.
\end{itemize}

\subsubsection*{Mitigation Strategies}
To mitigate supply chain attacks, organizations can implement several strategies:
\begin{itemize}
    \item \textbf{Remove unnecessary dependencies}: regularly review and audit dependencies to remove any that are not essential;
    \item \textbf{Use trusted sources}: only use dependencies from reputable sources and verify their integrity;
    \item \textbf{Inventory management}: maintain an up-to-date SBOM to track all components and dependencies used in the software;
    \item \textbf{Monitor for vulnerabilities}: use automated tools to monitor dependencies for known vulnerabilities;
    \item \textbf{Monitor unmaintaned dependencies}: be cautious when using dependencies that are no longer actively maintained. Consider finding alternatives or forking and maintaining them internally;
\end{itemize}

