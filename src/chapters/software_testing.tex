\section{Security testing}
Software analysis and testing are applied to detect security issues. 
These are usually driven by the programmer's experience to recognize patterns and situations in the code. 
Its automation is a key aspect if we want to build secure software projects.

\subsection{Verification and Validation}
Software verification and validation are two procedures needed for checking that a software system meets specifications and fulfills its intended purpose.
\begin{itemize}
    \item \textbf{Verification}: Are we building the product right?
    \item \textbf{Validation}: Are we building the right product?
\end{itemize}
More in detail, we can define:
\begin{itemize}
    \item \textbf{Verification}: the process of evaluating software to determine whether the products of a given development phase satisfy the conditions imposed by the requirements.
    During this phase, we usually check for bugs in the application;
    \item \textbf{Validation}: the process of evaluating software during or at the end of the development process to determine whether it satisfies specified requirements.
\end{itemize}

\subsection{Software Verification techniques}
Software verification techniques can be classified into two main categories: \textit{dynamic analysis} and \textit{static analysis}.

\subsubsection{Dynamic analysis (or Testing)}
Dynamic analysis involves exercising the software and observing the behavior and the produced output.
\begin{itemize}
    \item Test data are needed to execute the software;
    \item Can reveal the presence of errors, not their absence. In fact, only a subset of possible executions can be tested.
\end{itemize}
To conduct Dynamic analysis, firstly test cases must be identified to test the software under stress.
\begin{itemize}
    \item They are based on specifically defined input data. We will see later how to generate them to cover as many scenarios as possible;
    \item An oracle that describes what the system is supposed to do is needed. It will be necessary to check program results.
\end{itemize}
\begin{figure}[H]
    \centering
    \includegraphics[width=0.9\textwidth]{../../images/dynamic_analysis.png}
    \caption{Dynamic analysis process}
\end{figure}

\subsubsection{Static analysis}
Static analysis examines the software’s source code without executing the program.
In this context, the oracle is represented by a set of rules or properties that the code must satisfy. 
It is general and fixed.
Compared to dynamic analysis, static analysis has the following characteristics:
\begin{itemize}
    \item It can demonstrate the absence of specific classes of errors;
    \item It can analyze all possible executions of the program;
    \item Doesn't require test cases or execution of the program;
\end{itemize}
\begin{figure}[H]
    \centering
    \includegraphics[width=0.7\textwidth]{../../images/static_analysis.png}
    \caption{Static analysis process}
\end{figure}