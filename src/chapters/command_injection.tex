\section{Command Injection}
\textit{Command Injection} is an attack in which an attacker's goal is the arbitrary execution of commands on the host operating system.
It is a type of injection attack, in which no direct code injection occurs, but rather the injected data is used to compose commands that are executed by the system.

\paragraph{Example:} An application takes a parameter as input from a user and uses it to construct a command:
\begin{lstlisting}[language=PHP]
<?php
    //http://yourdomain.com?par=value
    $command = "ls " . $_GET['par'];
    $output = exec($command);
?>
\end{lstlisting}
An attacker could exploit this by providing a value such as \texttt{; rm -rf /}, resulting in the execution of both the \texttt{ls} command 
and the \texttt{rm -rf /} command, which would delete all files on the server.

\paragraph{Problem:} Untrusted external data is directly passed to a command interpreter. 
If the data is crafted to include commands, those commands may be executed and the interpreter may be forced to perform actions beyond its intended function.

\paragraph{Underlying conditions:} For command injection to succeed, the application typically meets three main conditions:
\begin{itemize}
    \item the application has the privileges/permissions to execute system commands;
    \item the application uses user-provided data as part of a system command;
    \item the user-provided data is not properly escaped or sanitized before use.
\end{itemize}

\paragraph{Mitigation:}
\begin{itemize}
    \item Avoid invoking shell/OS execution functions when possible;
    \item Treat all external input as untrusted: validate, normalize, and apply whitelisting where feasible;
    \item Use parameterized APIs or pass user input as separate arguments rather than concatenating it into command strings.
    \item Run the application with the least privileges necessary to reduce the impact of a successful attack.
\end{itemize}