\section{Error Handling}
Error handling is a mechanism used to detect and manage errors that occur during the execution of a program. It allows the application to recover gracefully from unexpected conditions and avoid crashes. Proper error handling is an essential part of an application's overall security.

\paragraph{Problem:}
Improper error handling occurs when software fails to correctly handle certain error conditions, leaving the program in an insecure or unstable state.  
This can lead to program crashes (potentially causing a \textbf{Denial of Service}) or unintentional disclosure of sensitive information such as stack traces, file paths, or internal logic — all of which may help an attacker gather useful insights about the system.

\paragraph{Mitigation:}
\begin{itemize}
    \item Always handle errors and exceptions appropriately, considering the specific context of each case;
    \item Avoid displaying detailed error messages to end users — log them securely instead;
    \item Do not silently ignore or mask exceptions, as this can hide critical failures and lead to unpredictable behavior;
    \item Always check the return value or status code of functions and system calls for possible errors;
\end{itemize}