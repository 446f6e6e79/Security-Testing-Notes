\section{Insecure Direct Object Reference (IDOR)}
\textit{Insecure Direct Object Reference} (IDOR) is a type of \textit{access control} vulnerability that occurs when 
an application uses user-supplied input to access internal objects directly, without proper authorization checks.  
This happens when a developer exposes a reference to internal implementation objects — such as files, database records, 
or keys — without adequate validation, allowing attackers to manipulate those references and gain unauthorized access 
to data or functionality.

\paragraph{Example:}
Consider a web application that allows users to view their profile information by accessing a URL like:
\begin{verbatim}
https://example.com/user/profile?user_id=123
\end{verbatim}
If authorization checks are not implemented, an attacker could change the \texttt{user\_id} parameter to another user's ID (e.g., \texttt{user\_id=124}) 
and gain access to that user's profile information.

\paragraph{How:}
\begin{itemize}
    \item \textbf{URL tampering:} modifying the value of a parameter directly in the browser's address bar (e.g., changing \texttt{user\_id=123} to \texttt{user\_id=124});
    \item \textbf{Body manipulation:} similar to URL tampering, but the attacker modifies parameters in the request body (e.g., in POST or PUT requests);
    \item \textbf{Path traversal:} already discussed in Section~\ref{Dir-Traversal};
    \item \textbf{Cookie or JSON ID manipulation:} if the application is vulnerable to IDOR, an attacker can alter identifiers stored in cookies or JSON payloads to access other users' data.
\end{itemize}